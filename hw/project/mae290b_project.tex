\documentclass[11pt]{article}

%% WRY has commented out some unused packages %%
%% If needed, activate these by uncommenting
\usepackage{geometry}                % See geometry.pdf to learn the layout options. There are lots.
%\geometry{letterpaper}                   % ... or a4paper or a5paper or ... 
\geometry{a4paper,left=2.5cm,right=2.5cm,top=2.5cm,bottom=2.5cm}
%\geometry{landscape}                % Activate for rotated page geometry
%\usepackage[parfill]{parskip}    % Activate to begin paragraphs with an empty line rather than an indent

%for figures
%\usepackage{graphicx}

\usepackage{color}
\definecolor{mygreen}{RGB}{28,172,0} % color values Red, Green, Blue
\definecolor{mylilas}{RGB}{170,55,241}
%% for graphics this one is also OK:
\usepackage{epsfig}

%% AMS mathsymbols are enabled with
\usepackage{amssymb,amsmath}

%% more options in enumerate
\usepackage{enumerate}
\usepackage{enumitem}

%% insert code
\usepackage{listings}

\usepackage[utf8]{inputenc}

\usepackage{hyperref}


% Default fixed font does not support bold face
\DeclareFixedFont{\ttb}{T1}{txtt}{bx}{n}{12} % for bold
\DeclareFixedFont{\ttm}{T1}{txtt}{m}{n}{12}  % for normal

% Custom colors
\usepackage{color}
\definecolor{deepblue}{rgb}{0,0,0.5}
\definecolor{deepred}{rgb}{0.6,0,0}
\definecolor{deepgreen}{rgb}{0,0.5,0}


% Python style for highlighting
\newcommand\pythonstyle{\lstset{
language=Python,
basicstyle=\ttm,
otherkeywords={self},             % Add keywords here
keywordstyle=\ttb\color{deepblue},
emph={MyClass,__init__},          % Custom highlighting
emphstyle=\ttb\color{deepred},    % Custom highlighting style
stringstyle=\color{deepgreen},
frame=tb,                         % Any extra options here
showstringspaces=false            % 
}}

% Python environment
\lstnewenvironment{python}[1][]
{
\pythonstyle
\lstset{#1}
}
{}

% Python for external files
\newcommand\pythonexternal[2][]{{
\pythonstyle
\lstinputlisting[#1]{#2}}}

% Python for inline
\newcommand\pythoninline[1]{{\pythonstyle\lstinline!#1!}}

%\usepackage{epstopdf}
%\DeclareGraphicsRule{.tif}{png}{.png}{`convert #1 `dirname #1`/`basename #1 .tif`.png}

%% To save typing, create some shortcuts
\newcommand{\ord}{\mbox{ord}}
\newcommand{\Ai}{\mbox{Ai}}
\newcommand{\Bi}{\mbox{Bi}}
\newcommand{\half}{\tfrac{1}{2}}
\newcommand{\defn}{\stackrel{\text{def}}{=}}
%% Use Roman font for special numbers and differentials:
\newcommand{\ii}{\mathrm{i}}
\newcommand{\dd}{\mathrm{d}}
\newcommand{\ee}{\mathrm{e}}
\newcommand{\su}{\mathsf{u}}
\newcommand{\sv}{\mathsf{v}}

\newcommand{\com}{\, ,}
\newcommand{\per}{\, .}

\newcommand{\noi}{\noindent}

\def\beq{\begin{equation}}
\def\eeq{\end{equation}}


%% stop typing all of epsilon and delta
\newcommand{\ep}{\ensuremath {\epsilon}}
\newcommand{\de}{\ensuremath {\delta}}

%% colors
\usepackage{graphicx,xcolor,lipsum}


\usepackage{mathtools}

\usepackage{graphicx}
\newcommand*{\matminus}{%
  \leavevmode
  \hphantom{0}%
  \llap{%
    \settowidth{\dimen0 }{$0$}%
    \resizebox{1.1\dimen0 }{\height}{$-$}%
  }%
}


\title{MAE290B, Final project\footnote{This project was developed using a Jupyter Notebook. An html version of the notebook is available at \href{https://github.com/crocha700/mae290b}{https://github.com/crocha700/mae290b}}}
\author{Cesar B Rocha}
\date{\today}

\begin{document}
\maketitle

We solve the two-dimensional diffusion equation 

\beq
\label{exact_eqn}
T_t = \alpha\left(T_{xx}+T_{yy}\right) + Q(x,y)\com
\eeq
 in a rectangular domain. The boundary conditions are


\beq
\label{bc_eqn}
T(x,y=0,t) = 4\sin\left(2\pi k\,x/l_x\right)\qquad \text{and}\qquad T(x,y=l_y,t) = 0\per
\eeq
The initial condition is

\beq
\label{ic_eqn}
T(x,y,t=0) = 0, \,y\neq0\per
\eeq
$Q$ is the source

\beq
\label{ic_eqn}
Q(x,y) = q(y)\,\sin\left(2\pi k\,x/l_x\right)\, \text{where}\qquad q_0\,\ee^{-a(y-l_y/2)^2}\com
\eeq


\begin{enumerate}[label=(\alph*)]
    \item Because \eqref{exact_eqn} is linear and the  boundary conditions and the source are periodic in $x$ the exact solution is separable in $x-(y,t)$:

        \beq
        \label{separable_soln}
        T(x,y,t) = \sin\left(2\pi\,k\,x/l_x\right)\,f(y,t).
        \eeq
        On inserting \eqref{separable_soln} in \eqref{exact_eqn} we obtain an partial differential equation for $f$:

        \beq
        \label{f_eqn}
    \partial_t f = \alpha\left(-\frac{4\pi^2k^2}{l_x^2}\,f + \partial^2_{yy}f\right) + q(y)\per
        \eeq
        The boundary conditions in terms of $f$ are

        \beq
        \label{f_bc}
            f(y=0,t) = 4\qquad \text{and}\qquad f(y=l_y,t) = 0\per
        \eeq
        The initial condition is
        \beq
            f(y=0,t) = 0, \,y\neq0\per\per
        \eeq



        It is important to realize that \eqref{f_eqn} is an exact result that follows from the linearity of the governing equation and the simplicity of the source and boundary conditions. The finite-differences discretization on a mesh of resolution $\Delta x$ leads to modified wavenumber, $k'= 4\sin^2\left(\tfrac{2\pi}{l_x}\,\kappa\Delta x/2\right)$, where $\kappa$ is the range of wavenumbers resolved by the discrete the grid: $0\leq\tfrac{\kappa\Delta x}{l_x}\leq 1/2$. For $\kappa = k$, the solution only produces very accurate result if $\tfrac{k\Delta x}{l_x}<<1$ (i.e. the grid spacing is very small compared to the wavelength of the boundary condition and source).

    \item Discretizing \eqref{f_eqn} in $y$ using a second-order centered finite-differences scheme, we obtain

        \beq
            \label{f_eqn_discrete}
            \partial_t\,f_j = \beta f_{j+1} - \left(2\beta + \gamma\right)f_j + \beta f_{j-1} + q_j\com
        \eeq
        where

        \beq
        \label{beta_gamma_defn}
        \beta \defn \frac{\alpha}{\Delta y^2} \qquad\text{and}\qquad \gamma \defn \frac{4\pi^2\alpha k^2}{l_x^2}\com
        \eeq
        \beq
        q_j = q_0\,\ee^{-a(j\Delta y - l_y/2)^2}\com
        \eeq
        and $j=1,2,\ldots,N-1$ are the interior nodes indices. Equation \eqref{f_eqn_discrete} is system of $N-1$ ordinary differential equations. We march this system using a third-order Runge-Kutta scheme. To estimate the optimum time-step to ensure stability, we employ a modified wavenumber analysis. For the purpose of this analysis, we assume solutions in the form
        \beq
        f(y,t) = \hat{f}\,\ee^{\ii(2\pi\,l\,y/l_y)}\com
        \eeq
        and consider the homogeneous (sourceless) problem. We obtain

        \beq
        \partial_t\,\hat f = - \alpha\left[\frac{4\pi^2k^2}{l_x^2} + \frac{4}{\Delta y^2}\sin^2\left(\tfrac{2\pi\,l}{l_y}\tfrac{\Delta y}{2}\right)\right] \hat f \per
        \eeq
        From Homework Assignment 1 we have that for real eigenvalues the third-order Runge-Kutta scheme requires

        \beq
             \alpha\left[\frac{4\pi^2k^2}{l_x^2} + \frac{4}{\Delta y^2}\sin^2\left(\tfrac{2\pi\,l}{l_y}\tfrac{\Delta y}{2}\right)\right] \Delta t \leq 2.5127\per
        \eeq
        The most restrictive scenario gives 
        \beq
        \label{stability_rk3}
        \Delta t \leq \frac{2.5127}{\alpha\left[\frac{4\pi^2k^2}{l_x^2} + \frac{4}{\Delta y^2}\right] }\per
        \eeq

        We code a simple function to integrate \eqref{f_eqn_discrete} using $N=11$ and the following parameters

        \begin{align}
            l_x = 2& \qquad\text{and}\qquad l_y=2\com \nonumber\\
            \alpha = 2 & \qquad\text{and}\qquad a=10 \qquad \text{and}\qquad k=1\per
        \end{align}


        The solution is integrated until it converges to steady state within machine double precision. 
        
        Before we proceed with the calculations it is useful to the test the code developed. To do that we note that a simple exact solution to the steady version of \eqref{f_eqn}-\eqref{f_bc} is available:

        \beq
            \label{f_bcc}
            f_e  = \frac{4}{\sinh{\left(2\pi\,k\tfrac{l_y}{l_x}\right)}}\,\sinh{\left[2\pi\,k\tfrac{\left(l_y-y\right)}{l_x}\right]}
        \eeq


        Figure \ref{comp_exact} show a comparison of the exact solution, $f_e$, against the numerical one computed using RK3. Clearly the numerical solution converges to the correct solution. The error in the numerical solution, measured by the root-mean-square of the difference to the exact solution, is $\sim 0.008$. This error is essentially due to the spatial discretization in y-direction.

        \begin{figure}[p]
        \centerline{\epsfig{file=f_rk3_exact.eps,width=12cm}}
        \caption{Testing the code: A comparison of exact and numerical steady solutions to  \eqref{f_eqn}-\eqref{f_bc} without source ($q_0$).}
        \label{comp_exact}
        \end{figure}


        \label{separable_soln}
        We integrate equation \eqref{f_eqn} using with different source strengths $q_0=-0.04,0,0.04$. With such a coarse resolution in $y$ the stability criterion \eqref{stability_rk3}  gives a very-large time-step $\sim 2$. To ensure good accuracy we choose $\Delta t = 1.$. It takes about 130 iterations for the solution to converge to steady-state within machine precision. The solution for $f$ is then multiplied by the sine function to form the solution for $T$ \eqref{separable_soln}. A slice of the solutions with different source strengths is shown in Figure \ref{soln_rk3}. The magnitude of the periodic solution in $x$ is larger for the $q_0=0.04$ case. For the  sink (negative source) case, the solution has smaller magnitude and opposite phase to the source. The solution with no source has also small magnitude compared to the source solution.



        \begin{figure}[p]
        \centerline{\epsfig{file=T_rk3.eps,width=12cm}}
        \caption{Slice of the RK3 numerical solution at $y=l_y/2$ for three different source strengths.}
        \label{soln_rk3}
        \end{figure}

    \item Using Crank-Nicolson, the time discretization of \eqref{f_eqn_discrete} is

        \beq
            \label{f_eqn_discrete_cn}
            \frac{f_i^{n+1}-f_i^n}{\Delta t} = \frac{1}{2}\left(\mathsf{D}^{n}f_j+\mathsf{D}^{n+1}f_j\right) + q_j\com
        \eeq
        where

        \beq
            \label{D_defn}
            \mathsf{D}f_j  \defn   \beta f_{j+1} - \left(2\beta + \gamma\right)f_j + \beta f_{j-1}\per
        \eeq
        with $\beta$ and $\gamma$ defined in \eqref{beta_gamma_defn}. Rearranging terms we obtain

        \begin{align}
            \label{tridiag_general}
            -\tfrac{\Delta t}{2}\beta\,f_{j-1}^{n+1} & + \left(1 + \Delta t \beta + \tfrac{\gamma \Delta t}{2}\right)\,f_j^{n+1} -\tfrac{\Delta t}{2}\beta\,f_{j+1}^{n+1} =   \nonumber \\ &\tfrac{\Delta t}{2}\beta\,f_{j-1}^{n}+\left(1 - \Delta t \beta - \tfrac{\gamma \Delta t}{2}\right)\,f_j^{n} + \tfrac{\Delta t}{2}\beta\,f_{j+1}^{n} + q_j
        \end{align}
        The right-hand-side of \eqref{tridiag_general} is known. Hence this is a simple $(N-1)\times(N-1)$ tridiagonal system. Applying the boundary condition we obtain, at node $j=1$,
        \begin{align}
            \label{tridiag_general_1}
            &  \left(1 + \Delta t \beta + \tfrac{\gamma \Delta t}{2}\right)\,f_1^{n+1} -\tfrac{\Delta t}{2}\beta\,f_{2}^{n+1} =   \nonumber \\ &4\Delta t\beta+\left(1 - \Delta t \beta - \tfrac{\gamma \Delta t}{2}\right)\,f_1^{n} + \tfrac{\Delta t}{2}\beta\,f_{2}^{n} + q_1\per
        \end{align}
        Similarly, at node $j=N-1$,
          \begin{align}
            \label{tridiag_general_N1}
            -\tfrac{\Delta t}{2}\beta\,f_{N-1}^{n+1} & + \left(1 + \Delta t \beta + \tfrac{\gamma \Delta t}{2}\right)\,f_{N-1}^{n+1} =   \nonumber \\ &\tfrac{\Delta t}{2}\beta\,f_{N-2}^{n}+\left(1 - \Delta t \beta - \tfrac{\gamma \Delta t}{2}\right)\,f_{N-1}^{n}+ q_{N-1}\per
        \end{align}


        To implement the Crank-Nicolson solver we first code a simple function to solve the tridiagonal linear system using Thomas' algorithm. The system of ODEs is then step forward until the solution converges to steady states within machine precision. The Crank-Nicolson scheme is unconditionally stable. To ensure good accuracy we set $\Delta t = 1$. The solution converges after 230 iterations. Notice that this is the same number of iterations as the RK3 scheme. With better resolution in $y$, however, the RK3 scheme would be constrained to smaller time-steps (proportional to $\Delta y^2$) while the CN would be stable stable.  Figure \ref{soln_cn} show a slice of the steady-state solution at $y=l_y/2$. The solution is virtually the same as the RK3 solution discussed above.


        \begin{figure}[p]
        \centerline{\epsfig{file=Tcn_y1.eps,width=12cm}}
        \caption{Slice of the CN numerical solution at $y=l_y/2$ for three different source strengths.}
        \label{soln_cn}
        \end{figure}


\end{enumerate}

%\maketitle
%\begin{figure}[p]
%\centerline{\epsfig{file=sin_four.png,width=12cm}}
%\caption{Initial guess and equilibrated solution for Laplace's equation in a square box. The initial guess is $\phi_0 = \sin 4\pi x + \sin 4\pi y$.}
%    \label{sin_four}
%\end{figure}






\end{document}


