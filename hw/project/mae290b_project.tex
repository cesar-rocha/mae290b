\documentclass[11pt]{article}

%% WRY has commented out some unused packages %%
%% If needed, activate these by uncommenting
\usepackage{geometry}                % See geometry.pdf to learn the layout options. There are lots.
%\geometry{letterpaper}                   % ... or a4paper or a5paper or ... 
\geometry{a4paper,left=2.5cm,right=2.5cm,top=2.5cm,bottom=2.5cm}
%\geometry{landscape}                % Activate for rotated page geometry
%\usepackage[parfill]{parskip}    % Activate to begin paragraphs with an empty line rather than an indent

%for figures
%\usepackage{graphicx}

\usepackage{color}
\definecolor{mygreen}{RGB}{28,172,0} % color values Red, Green, Blue
\definecolor{mylilas}{RGB}{170,55,241}
%% for graphics this one is also OK:
\usepackage{epsfig}

%% AMS mathsymbols are enabled with
\usepackage{amssymb,amsmath}

%% more options in enumerate
\usepackage{enumerate}
\usepackage{enumitem}

%% insert code
\usepackage{listings}

\usepackage[utf8]{inputenc}

\usepackage{hyperref}


% Default fixed font does not support bold face
\DeclareFixedFont{\ttb}{T1}{txtt}{bx}{n}{12} % for bold
\DeclareFixedFont{\ttm}{T1}{txtt}{m}{n}{12}  % for normal

% Custom colors
\usepackage{color}
\definecolor{deepblue}{rgb}{0,0,0.5}
\definecolor{deepred}{rgb}{0.6,0,0}
\definecolor{deepgreen}{rgb}{0,0.5,0}


% Python style for highlighting
\newcommand\pythonstyle{\lstset{
language=Python,
basicstyle=\ttm,
otherkeywords={self},             % Add keywords here
keywordstyle=\ttb\color{deepblue},
emph={MyClass,__init__},          % Custom highlighting
emphstyle=\ttb\color{deepred},    % Custom highlighting style
stringstyle=\color{deepgreen},
frame=tb,                         % Any extra options here
showstringspaces=false            % 
}}

% Python environment
\lstnewenvironment{python}[1][]
{
\pythonstyle
\lstset{#1}
}
{}

% Python for external files
\newcommand\pythonexternal[2][]{{
\pythonstyle
\lstinputlisting[#1]{#2}}}

% Python for inline
\newcommand\pythoninline[1]{{\pythonstyle\lstinline!#1!}}

%\usepackage{epstopdf}
%\DeclareGraphicsRule{.tif}{png}{.png}{`convert #1 `dirname #1`/`basename #1 .tif`.png}

%% To save typing, create some shortcuts
\newcommand{\ord}{\mbox{ord}}
\newcommand{\Ai}{\mbox{Ai}}
\newcommand{\Bi}{\mbox{Bi}}
\newcommand{\half}{\tfrac{1}{2}}
\newcommand{\defn}{\stackrel{\text{def}}{=}}
%% Use Roman font for special numbers and differentials:
\newcommand{\ii}{\mathrm{i}}
\newcommand{\dd}{\mathrm{d}}
\newcommand{\ee}{\mathrm{e}}
\newcommand{\su}{\mathsf{u}}
\newcommand{\sv}{\mathsf{v}}

\newcommand{\com}{\, ,}
\newcommand{\per}{\, .}

\newcommand{\noi}{\noindent}

\def\beq{\begin{equation}}
\def\eeq{\end{equation}}


%% stop typing all of epsilon and delta
\newcommand{\ep}{\ensuremath {\epsilon}}
\newcommand{\de}{\ensuremath {\delta}}

%% colors
\usepackage{graphicx,xcolor,lipsum}


\usepackage{mathtools}

\usepackage{graphicx}
\newcommand*{\matminus}{%
  \leavevmode
  \hphantom{0}%
  \llap{%
    \settowidth{\dimen0 }{$0$}%
    \resizebox{1.1\dimen0 }{\height}{$-$}%
  }%
}


\title{MAE290B, Final Project\footnote{This project was developed using a Jupyter Notebook. An html version of the notebook is available at \href{https://github.com/crocha700/mae290b}{https://github.com/crocha700/mae290b}}}
\author{Cesar B Rocha}
\date{\today}

\begin{document}
\maketitle

\section*{Abstract}

We solve the two-dimensional diffusion equation with source in a square domain periodic in the $x-$direction, with particular focus on the steady-state solution. With coarse resolution in the $y-$direction there is no notable difference in performance between a third-order Runge-Kutta  and a Crank-Nicolson (CN) schemes for time marching. With higher resolution, however, the CN scheme requires much less iterations to converge because it is free of stability restrictions on the time-step. We show that decreasing the diffusion coefficient proportionally increases the time necessary to achieve steady-state. We verify that the spatial discretization scheme is second-order accurate by performing experiments with a hierarchy of finner grids in the $y-$direction and comparing the numerical solutions for the sourceless case ($q_0=0$) with the exact solution.

\section*{Main text}

\noindent We consider the two-dimensional diffusion equation 

\beq
\label{exact_eqn}
T_t = \alpha\left(T_{xx}+T_{yy}\right) + Q(x,y)\com
\eeq
in a rectangular domain of size $l_x$ and $l_y$ in the $x-$ an $y-$direction, respectively. The boundary conditions are


\beq
\label{bc_eqn}
T(x,y=0,t) = 4\sin\left(2\pi k\,x/l_x\right)\qquad \text{and}\qquad T(x,y=l_y,t) = 0\per
\eeq
The initial condition is

\beq
\label{ic_eqn}
T(x,y,t=0) = 0, \,y\neq0\per
\eeq
$Q$ is the source

\beq
\label{ic_eqn}
Q(x,y) = q(y)\,\sin\left(2\pi k\,x/l_x\right)\,,\, \text{where}\qquad q_0\,\ee^{-a(y-l_y/2)^2}\com
\eeq


\begin{enumerate}[label=(\alph*)]
    \item Because \eqref{exact_eqn} is linear and the  boundary conditions and the source are periodic in $x$ the exact solution is separable in $x-(y,t)$:

        \beq
        \label{separable_soln}
        T(x,y,t) = \sin\left(2\pi\,k\,x/l_x\right)\,f(y,t).
        \eeq
        On inserting \eqref{separable_soln} in \eqref{exact_eqn} we obtain an partial differential equation for $f$:

        \beq
        \label{f_eqn}
    \partial_t f = \alpha\left(-\frac{4\pi^2k^2}{l_x^2}\,f + \partial^2_{yy}f\right) + q(y)\per
        \eeq
        The boundary conditions in terms of $f$ are

        \beq
        \label{f_bc}
            f(y=0,t) = 4\qquad \text{and}\qquad f(y=l_y,t) = 0\per
        \eeq
        The initial condition is
        \beq
            f(y=0,t) = 0, \,y\neq0\per\per
        \eeq



        It is important to realize that \eqref{f_eqn} is an exact result that follows from the linearity of the governing equation and the simplicity of the source and boundary conditions. The finite-differences discretization on a mesh of resolution $\Delta x$ leads to modified wavenumber, $k'= 4\sin^2\left(\tfrac{2\pi}{l_x}\,\kappa\Delta x/2\right)$, where $\kappa$ is the range of wavenumbers resolved by the discrete the grid: $0\leq\tfrac{\kappa\Delta x}{l_x}\leq 1/2$. For $\kappa = k$, the solution only produces very accurate result if $\tfrac{k\Delta x}{l_x}<<1$ (i.e. the grid spacing is very small compared to the wavelength of the boundary condition and source).

    \item Discretizing \eqref{f_eqn} in $y$ using a second-order centered finite-differences scheme, we obtain

        \beq
            \label{f_eqn_discrete}
            \partial_t\,f_j = \beta f_{j+1} - \left(2\beta + \gamma\right)f_j + \beta f_{j-1} + q_j\com
        \eeq
        where

        \beq
        \label{beta_gamma_defn}
        \beta \defn \frac{\alpha}{\Delta y^2} \qquad\text{and}\qquad \gamma \defn \frac{4\pi^2\alpha k^2}{l_x^2}\com
        \eeq
        \beq
        q_j = q_0\,\ee^{-a(j\Delta y - l_y/2)^2}\com
        \eeq
        and $j=1,2,\ldots,N-1$ are the interior nodes indices. Equation \eqref{f_eqn_discrete} is system of $N-1$ ordinary differential equations. We march this system using a third-order Runge-Kutta scheme. To estimate the optimum time-step to ensure stability, we employ a modified wavenumber analysis. For the purpose of this analysis, we assume solutions in the form
        \beq
        f(y,t) = \hat{f}\,\ee^{\ii(2\pi\,l\,y/l_y)}\com
        \eeq
        and consider the homogeneous (sourceless) problem. We obtain

        \beq
        \partial_t\,\hat f = - \alpha\left[\frac{4\pi^2k^2}{l_x^2} + \frac{4}{\Delta y^2}\sin^2\left(\tfrac{2\pi\,l}{l_y}\tfrac{\Delta y}{2}\right)\right] \hat f \per
        \eeq
        From Homework Assignment 1 we have that for real eigenvalues the third-order Runge-Kutta scheme requires

        \beq
             \alpha\left[\frac{4\pi^2k^2}{l_x^2} + \frac{4}{\Delta y^2}\sin^2\left(\tfrac{2\pi\,l}{l_y}\tfrac{\Delta y}{2}\right)\right] \Delta t \leq 2.5127\per
        \eeq
        The most restrictive scenario gives 
        \beq
        \label{stability_rk3}
        \Delta t \leq \frac{2.5127}{\alpha\left[\frac{4\pi^2k^2}{l_x^2} + \frac{4}{\Delta y^2}\right] }\per
        \eeq

        We code a simple function\footnote{All codes are submitted via TED.} to integrate \eqref{f_eqn_discrete} using $N=11$ and the following parameters

        \begin{align}
            l_x = 2& \qquad\text{and}\qquad l_y=2\com \nonumber\\
            \alpha = 2 & \qquad\text{and}\qquad a=10 \qquad \text{and}\qquad k=1\per
        \end{align}


        The solution is integrated until it converges to steady state within machine double precision. 
        
        Before we proceed with the calculations it is useful to the test the code developed. To do that we note that a simple exact solution to the steady version of \eqref{f_eqn}-\eqref{f_bc} is available:

        \beq
            \label{f_bcc}
            f_e  = \frac{4}{\sinh{\left(2\pi\,k\tfrac{l_y}{l_x}\right)}}\,\sinh{\left[2\pi\,k\tfrac{\left(l_y-y\right)}{l_x}\right]}
        \eeq


        Figure \ref{comp_exact} show a comparison of the exact solution, $f_e$, against the numerical one computed using RK3. Clearly the numerical solution converges to the correct solution. The error in the numerical solution, measured by the root-mean-square of the difference to the exact solution, is $\sim 0.008$. This error is essentially due to the spatial discretization in y-direction.

        \begin{figure}[p]
        \centerline{\epsfig{file=f_rk3_exact.eps,width=12cm}}
        \caption{Testing the code: A comparison of exact and numerical steady solutions to  \eqref{f_eqn}-\eqref{f_bc} without source ($q_0$).}
        \label{comp_exact}
        \end{figure}


        \label{separable_soln}
        We integrate equation \eqref{f_eqn} using with different source strengths $q_0=-0.04,0,0.04$. With such a coarse resolution in $y$ the stability criterion \eqref{stability_rk3}  gives a very-large time-step $\sim 2$. To ensure good accuracy we choose $\Delta t = 1.$. It takes about 130 iterations for the solution to converge to steady-state within machine precision. The solution for $f$ is then multiplied by the sine function to form the solution for $T$ \eqref{separable_soln}. A slice of the solutions with different source strengths is shown in Figure \ref{soln_rk3}. The magnitude of the periodic solution in $x$ is larger for the $q_0=0.04$ case. For the  sink (negative source) case, the solution has smaller magnitude and opposite phase to the source case. The solution with no source has also small magnitude compared to the source solution.

        \begin{figure}[p]
        \centerline{\epsfig{file=T_rk3.eps,width=12cm}}
        \caption{Slice of the RK3 numerical solution at $y=l_y/2$ for three different source strengths.}
        \label{soln_rk3}
        \end{figure}

    \item Using Crank-Nicolson, the time discretization of \eqref{f_eqn_discrete} is

        \beq
            \label{f_eqn_discrete_cn}
            \frac{f_i^{n+1}-f_i^n}{\Delta t} = \frac{1}{2}\left(\mathsf{D}^{n}f_j+\mathsf{D}^{n+1}f_j\right) + q_j\com
        \eeq
        where

        \beq
            \label{D_defn}
            \mathsf{D}f_j  \defn   \beta f_{j+1} - \left(2\beta + \gamma\right)f_j + \beta f_{j-1}\per
        \eeq
        with $\beta$ and $\gamma$ defined in \eqref{beta_gamma_defn}. Rearranging terms we obtain

        \begin{align}
            \label{tridiag_general}
            -\tfrac{\Delta t}{2}\beta\,f_{j-1}^{n+1} & + \left(1 + \Delta t \beta + \tfrac{\gamma \Delta t}{2}\right)\,f_j^{n+1} -\tfrac{\Delta t}{2}\beta\,f_{j+1}^{n+1} =   \nonumber \\ &\tfrac{\Delta t}{2}\beta\,f_{j-1}^{n}+\left(1 - \Delta t \beta - \tfrac{\gamma \Delta t}{2}\right)\,f_j^{n} + \tfrac{\Delta t}{2}\beta\,f_{j+1}^{n} + q_j
        \end{align}
        The right-hand-side of \eqref{tridiag_general} is known. Hence this is a simple $(N-1)\times(N-1)$ tridiagonal system. Applying the boundary condition we obtain, at node $j=1$,
        \begin{align}
            \label{tridiag_general_1}
            &  \left(1 + \Delta t \beta + \tfrac{\gamma \Delta t}{2}\right)\,f_1^{n+1} -\tfrac{\Delta t}{2}\beta\,f_{2}^{n+1} =   \nonumber \\ &4\Delta t\beta+\left(1 - \Delta t \beta - \tfrac{\gamma \Delta t}{2}\right)\,f_1^{n} + \tfrac{\Delta t}{2}\beta\,f_{2}^{n} + q_1\per
        \end{align}
        Similarly, at node $j=N-1$,
          \begin{align}
            \label{tridiag_general_N1}
            -\tfrac{\Delta t}{2}\beta\,f_{N-1}^{n+1} & + \left(1 + \Delta t \beta + \tfrac{\gamma \Delta t}{2}\right)\,f_{N-1}^{n+1} =   \nonumber \\ &\tfrac{\Delta t}{2}\beta\,f_{N-2}^{n}+\left(1 - \Delta t \beta - \tfrac{\gamma \Delta t}{2}\right)\,f_{N-1}^{n}+ q_{N-1}\per
        \end{align}


        To implement the Crank-Nicolson solver we first code a simple function to solve the tridiagonal linear system using Thomas' algorithm (Matrix set-up and code details in the Appendix). The system of ODEs is then step forward until the solution converges to steady states within machine precision. The Crank-Nicolson scheme is unconditionally stable. To ensure good accuracy we set $\Delta t = 1$. The solution converges after 230 iterations. Notice that this is the same number of iterations as the RK3 scheme.  Hence with coarse resolution in $y$, the RK3 schemme is preferable because it is third-order accurate as opposed to second-order accurace CN. With better resolution in $y$, however, the RK3 scheme would be constrained to smaller time-steps (proportional to $\Delta y^2$) while the CN would be stable stable with $\Delta t =1$.  Figure \ref{soln_cn} shows a slice of the steady-state solution at $y=l_y/2$. The solution is virtually the same computational cost as the RK3 solution discussed above. 


        \begin{figure}[p]
        \centerline{\epsfig{file=Tcn_y1.eps,width=12cm}}
        \caption{Slice of the CN numerical solution at $y=l_y/2$ for three different source strengths.}
        \label{soln_cn}
        \end{figure}



    \item We repeat the CN calculations in (c) with $\alpha = 0.0001$. First, to better understand the effect of the size of $\alpha$ on the convergence of the solution we consider a simplified version of \eqref{f_eqn} assuming that the horizontal scale of variation of $f$ in x is much smaller than the scale of variation in $y$, $\delta y$. That is we assume

        \beq
            k >> \frac{1}{\delta y}\per
        \eeq
        The general solution of this simplified equation has the  form

        \beq
            \label{general_soln}
            f_0(y,t) = f_0^H + f_0^I\per
        \eeq
        The homogeneous part $f_0^H$ corresponds to the transient solution, and it has the form

        \beq
        f_0^H(t) = C\,\ee^{-\tfrac{4\pi^2k^2}{l_x^2}\,\alpha t}\com
        \eeq
        where $C$ is a constant. The transient solution decays on a time-scale inversely proportional to $\alpha$. For fixed $k$ and $l_x$, reducing $\alpha$ implies that the steady state solution will be achieved on longer time-scales. Thus we expect that the convergence of the solution to steady-state will be about 100 slower  with $\alpha=0.0001$ than in previous experiments with $\alpha=0.01$. 

        Indeed it takes about 21863 iterations for the solution to converge to steady-state within machine double precision. Because we used the same time-step ($\Delta t =1$), this is about 100 slower than the previous calculations with a diffusion coefficient 100 bigger $\alpha = 0.01$. The physical interpretation is that it takes longer for source and boundary conditions to equilibrate by diffusion. Because the steady solution is inversely proportional to the diffusion coefficient the solution with non-zero source has also larger magnitude than the one with $\alpha=0.01$. 

        \begin{figure}[p]
        \centerline{\epsfig{file=Tcn2_y1.eps,width=9cm}\\
        \epsfig{file=Tcn2_green_y1.eps,width=9cm}}
        \caption{Slice of the CN $\alpha=0.0001$ numerical solution at $y=l_y/2$ for three different source strengths. The figure on the right panel shows the $q_0=0$ case in a zoomed-in $y-$axis.}
        \label{soln_cn}
        \end{figure}


    \item We refine the grid to explore the grid convergence. We focus on the solution using the CN scheme because, as discussed above, the solution using RK3 would become computationally costly for small $\Delta y$. In particular we perform experiments with different number of points in the $y-$direction $N = [5,10,50,100,200,300,400,500]$ with $\alpha = 0.01$ and $q_0=-0.04,0,0.04$.  The number of iterations necessary for convergence varies significantly from $230$ for $N=5$ to $N=17033$ for $N=500$ (Table 1).  The convergence is relatively fast: the solution with $N=50$ is visually indistinguishable from the solution with $N=500$. To assess the accuracy of  the discretization in the $y-$direction we can compare the solution with $q_0$ to the exact solution \eqref{f_bcc}. In particular we compute the relative error at $y=l_y/2$

        \beq
            \text{Error} = \frac{|f_{exact}(l_y/2)-f_{CN}(l_y/2)|}{|f_{exact}(l_y/2)|}\per
        \eeq
        Figure \ref{error_cn} shows the relative error at $y=l_y/2$ as a function of number of points in the $y-$direction. Clearly, the error decreases with $N$, following a $N^{-2}$ power law. This analysis thus validates that the scheme used for spatial discretization in the $y-$direction is second-order accurate.


    \begin{table}\label{iterations_conv}    
        \begin{center}
        \caption{Number of iterations necessary for convergence to steady-state of numerical solution using a CN scheme to march forward in time. The diffusion coefficient was $\alpha=0.01$. N is the number of points used for spatial discretization in the $y-$direction.}
        \begin{tabular}{ l |  c c c }\\
            $q_0$ & -0.04 & 0 & 0.04\\ \hline
            N & & & \\
             5 & 223 & 229 & 227\\
             10 & 225 & 229 & 231\\
             50 & 226 & 230 & 232\\
             100 & 701 & 701 & 701\\
             200 & 2765 & 2768 & 2766\\
             300 & 6184 & 6195 & 6176\\
             400 & 10949 & 10971 & 10963\\
             500 & 17168& 17006 & 17033 \\
        \end{tabular}
    \end{center}
        \end{table}


        \begin{figure}[h]
           \centerline{\epsfig{file=error_cn.eps,width=12cm}}
           \caption{The relative error at $y=l_y/2$ for sourceless ($q_0=0$) solution. For reference the $N^{-2}$ curve is plotted.}
        \label{error_cn}
        \end{figure}

        \begin{figure}[p]
        \centerline{\epsfig{file=f_various_N_a10_q0-4.eps,width=9cm}\\
        \epsfig{file=T_various_N_a10_q0-4.eps,width=9cm}}
        \caption{(Left panel) Steady-state solution for $y-$structure, $f$, and (right panel) slice of the solution for $T$ at $y=l_y/2$ for different resolutions in the $y-$direction. The source strength is $q_0=-0.04$.}
        \label{soln_cn_various_N_1}
        \end{figure}

       \begin{figure}[p]
        \centerline{\epsfig{file=f_various_N_a10_q00.eps,width=9cm}\\
        \epsfig{file=T_various_N_a10_q00.eps,width=9cm}}
        \caption{(Left panel) Steady-state solution for $y-$structure, $f$, and (right panel) slice of the solution for $T$ at $y=l_y/2$ for different resolutions in the $y-$direction.The source strength is $q_0=0$.}
        \label{soln_cn_various_N_2}
        \end{figure}

       \begin{figure}[p]
        \centerline{\epsfig{file=f_various_N_a10_q04.eps,width=9cm}\\
        \epsfig{file=T_various_N_a10_q04.eps,width=9cm}}
        \caption{(Left panel) Steady-state solution for $y-$structure, $f$, and (right panel) slice of the solution for $T$ at $y=l_y/2$ for different resolutions in the $y-$direction.The source strength is $q_0=0.04$.}
        \label{soln_cn_various_N_2}
        \end{figure}




\end{enumerate}


\newpage
\section*{Appendix: Matrix set-up for the Crank-Nicolson solution}

Application of the Crank-Nicolson scheme to march equation for $f$ in time leads to a simple tridiagonal system \eqref{tridiag_general}-\eqref{tridiag_general_1}-\eqref{tridiag_general_N1}. We can write that system in matrix form 

\beq
\label{lin_system}
\mathsf{T}\,\mathsf{f}^{n+1} = \mathsf{B\,}\mathsf{f}^{n} + \mathsf{b} + \mathsf{q}\com
\eeq
where $\mathsf{f} = [f_1,f_2,\ldots,f_N-2,f_N-1]$, $\mathsf{b}= [4\Delta\beta,0,\ldots,0]$, and $\mathsf{q} = [q_1,q_2,\ldots,q_N-2,q_N-1]$, and

\begin{equation} \label{matrixT}
    \mathsf{T} = \left[ \begin{matrix}
            \left(1 + \Delta t \beta + \tfrac{\gamma \Delta t}{2}\right) & -\tfrac{\Delta t}{2}\beta & &0\\
            -\tfrac{\Delta t}{2}\beta& \ddots & \ddots &  \vdots\\
                                     & \ddots & \left(1 + \Delta t \beta + \tfrac{\gamma \Delta t}{2}\right)  &-\tfrac{\Delta t}{2}\beta \\
                              0       &\ldots &-\tfrac{\Delta t}{2}\beta &\left(1 + \Delta t \beta + \tfrac{\gamma \Delta t}{2}\right)  \\
    \end{matrix} \right]\com
\end{equation}
and
\begin{equation} \label{matrixB}
    \mathsf{B} = \left[ \begin{matrix}
            \left(1 - \Delta t \beta - \tfrac{\gamma \Delta t}{2}\right) & \tfrac{\Delta t}{2}\beta & &0\\
            \tfrac{\Delta t}{2}\beta& \ddots & \ddots  & \vdots\\
                                     & \ddots & \left(1 + \Delta t \beta + \tfrac{\gamma \Delta t}{2}\right)  &\tfrac{\Delta t}{2}\beta \\
                              0       &\ldots &\tfrac{\Delta t}{2}\beta &\left(1 - \Delta t \beta - \tfrac{\gamma \Delta t}{2}\right)  \\
    \end{matrix} \right]\per
\end{equation}

Notice that we have represented the first part of the  know right-hand-side as a matrix multiplication but in practice there is no need to store all those zeros and perform the matrix multiplication. Instead, we vectorized the calculation in \texttt{python}. The linear system  \eqref{lin_system} is efficiently solved by implementing the Thomas algorithm in \texttt{python}.


\begin{lstlisting}[language=python]
def Thomas(a,b,c,s):
    ''' it solves T x = s where T is a tridiagonal matrix
            with constants coefficients

            Inputs
            ------------------------------------------
            a :: main diagonal coefficients
            b :: lower diagonals coefficients
            c :: upper diagonals coefficients
            s :: righ-hand side (input array)
            
            Outputs
            ------------------------------------------
            x :: solution array
    '''
    
    assert s.size == a.size, "s and a must have the same length"
    
    n = s.size
    
    rhs = s.copy()
    f = a.copy()                # assemble arrays: main diagonal
    e = b.copy()                #                  lower diagonal
    g = c.copy()                #                  upper diagonal
    x = np.empty(n)             #                  solution
                        
    for i in range(1,n):
        #e[i-1] = e[i-1]/f[i-1]  # compute and store the "factor"
        factor = e[i-1]/f[i-1]   # perform Gaussian elimination
        f[i] -=  factor*g[i-1]
        rhs[i] -=  factor*rhs[i-1]
        del factor
    
    x[-1] = rhs[-1]/f[-1]   # backward substitution to solve for x
    for i in range(n-2,-1,-1):
        x[i] = (rhs[i]-g[i]*x[i+1])/f[i]
        
    return x
\end{lstlisting}

\end{document}


