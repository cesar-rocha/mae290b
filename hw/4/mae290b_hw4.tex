\documentclass[11pt]{article}

%% WRY has commented out some unused packages %%
%% If needed, activate these by uncommenting
\usepackage{geometry}                % See geometry.pdf to learn the layout options. There are lots.
%\geometry{letterpaper}                   % ... or a4paper or a5paper or ... 
\geometry{a4paper,left=2.5cm,right=2.5cm,top=2.5cm,bottom=2.5cm}
%\geometry{landscape}                % Activate for rotated page geometry
%\usepackage[parfill]{parskip}    % Activate to begin paragraphs with an empty line rather than an indent

%for figures
%\usepackage{graphicx}

\usepackage{color}
\definecolor{mygreen}{RGB}{28,172,0} % color values Red, Green, Blue
\definecolor{mylilas}{RGB}{170,55,241}
%% for graphics this one is also OK:
\usepackage{epsfig}

%% AMS mathsymbols are enabled with
\usepackage{amssymb,amsmath}

%% more options in enumerate
\usepackage{enumerate}
\usepackage{enumitem}

%% insert code
\usepackage{listings}

\usepackage[utf8]{inputenc}

\usepackage{hyperref}


% Default fixed font does not support bold face
\DeclareFixedFont{\ttb}{T1}{txtt}{bx}{n}{12} % for bold
\DeclareFixedFont{\ttm}{T1}{txtt}{m}{n}{12}  % for normal

% Custom colors
\usepackage{color}
\definecolor{deepblue}{rgb}{0,0,0.5}
\definecolor{deepred}{rgb}{0.6,0,0}
\definecolor{deepgreen}{rgb}{0,0.5,0}


% Python style for highlighting
\newcommand\pythonstyle{\lstset{
language=Python,
basicstyle=\ttm,
otherkeywords={self},             % Add keywords here
keywordstyle=\ttb\color{deepblue},
emph={MyClass,__init__},          % Custom highlighting
emphstyle=\ttb\color{deepred},    % Custom highlighting style
stringstyle=\color{deepgreen},
frame=tb,                         % Any extra options here
showstringspaces=false            % 
}}

% Python environment
\lstnewenvironment{python}[1][]
{
\pythonstyle
\lstset{#1}
}
{}

% Python for external files
\newcommand\pythonexternal[2][]{{
\pythonstyle
\lstinputlisting[#1]{#2}}}

% Python for inline
\newcommand\pythoninline[1]{{\pythonstyle\lstinline!#1!}}

%\usepackage{epstopdf}
%\DeclareGraphicsRule{.tif}{png}{.png}{`convert #1 `dirname #1`/`basename #1 .tif`.png}

%% To save typing, create some shortcuts
\newcommand{\ord}{\mbox{ord}}
\newcommand{\Ai}{\mbox{Ai}}
\newcommand{\Bi}{\mbox{Bi}}
\newcommand{\half}{\tfrac{1}{2}}
\newcommand{\defn}{\stackrel{\text{def}}{=}}
%% Use Roman font for special numbers and differentials:
\newcommand{\ii}{\mathrm{i}}
\newcommand{\dd}{\mathrm{d}}
\newcommand{\ee}{\mathrm{e}}
\newcommand{\su}{\mathsf{u}}
\newcommand{\sv}{\mathsf{v}}

\newcommand{\com}{\, ,}
\newcommand{\per}{\, .}

\newcommand{\noi}{\noindent}

\def\beq{\begin{equation}}
\def\eeq{\end{equation}}


%% stop typing all of epsilon and delta
\newcommand{\ep}{\ensuremath {\epsilon}}
\newcommand{\de}{\ensuremath {\delta}}

%% colors
\usepackage{graphicx,xcolor,lipsum}


\usepackage{mathtools}

\usepackage{graphicx}
\newcommand*{\matminus}{%
  \leavevmode
  \hphantom{0}%
  \llap{%
    \settowidth{\dimen0 }{$0$}%
    \resizebox{1.1\dimen0 }{\height}{$-$}%
  }%
}


\title{MAE290B, Homework Assignment 3\footnote{This homework assignment was coded and developed using a Jupyter Notebook. An html version of the notebook is available at \href{https://github.com/crocha700/mae290b}{https://github.com/crocha700/mae290b}}}
\author{Cesar B Rocha}
%\date{October 16th, 2014}                                           
\date{\today}

\begin{document}
\maketitle

\section*{Problem 1}

\begin{enumerate}[label=(\alph*)]


    \item Using centered space approximation to discretize the spatial derivatives we obtain

        \beq
            \label{diff_discret_space}
            \partial_t \,T_i = \frac{\alpha}{\Delta x^2}\left(T_{i+1}-2T_i + T_{i+1}\right)\per
        \eeq
        We use a modified wavenumber analysis. Let $T = \hat{T}(t)\ee^{\ii k x}$. Taylor-expanding $T_{i+1}$ and $T_{i-1}$  about the $i$'th node, we obtain 
      
        \beq
            \label{diff_discret_space_mod}
            \partial_t \,\hat{T}_i = \frac{\alpha}{\Delta x^2}\left(\cos k\Delta x - 2\right) + \ord{(\Delta x^2)}\per
        \eeq
        The overall local discretization error is $\ord{(\Delta x^2)}$. Noticing that
        \beq
        \sin \left(k \tfrac{\Delta x}{2}\right)^2 = 2 - \cos k \Delta x\com
        \eeq
        the modified wavenumber is

        \beq
            \label{mod_k}
            k'^2 = \frac{4}{\Delta x^2}\sin \left(k \tfrac{\Delta x}{2}\right)^2\per
        \eeq
        The truncation error is $\ord{(\Delta x^2)}$. This represents the order of accuracy. The accuracy is clearly wavenumber dependent .For large scales ($k \Delta x$ small) we have

        Figure \ref{fig_mod_k} shows the modified wavenumber as a function of the wavenumber. A simple asymptotic analysis gives  $k'\to k$ as $k\Delta x \to 0$. That is the discretization highly accurate for small $k\Delta$. However, the accuracy is degraded at larger wavenumber. At $k\Delta=\pi$, $k'Delta x= 2$, which is a relative error of $\sim36\%$.
    
\begin{figure}[ht]
\centerline{\epsfig{file=modified_wavenumber_diff.png,width=12cm}}
\caption{Modified wavenumber analysis for second-order centered spatial discretization of the diffusion operator.}
    \label{fig_mod_k}
\end{figure}

        Equation \eqref{diff_discret_space} has the form of the model problem $y' = \lambda y$. From Homework Assignment 1 we have that the Crank-Nicolson stability condition requires
        \beq
            \label{cn_cond}
            \frac{|1+\lambda \Delta t|}{|1-\lambda \Delta t|} \leq 1\com
        \eeq
        which is always satisfied for $\lambda < 0$ (the scheme is unconditionally stable). Here we change lambda with $-k'^2$, which is always negative (this assumes $\alpha>0$; no negative diffusion here!). This implicit scheme is unconditionally stable.
    
        From the modified wavenumber expansion  \eqref{diff_discret_space_mod} we notice that the scheme is second-order in space. The Crank-Nicolson (CN) method for time stepping is also second-order. To see that, we notice that \eqref{diff_discret_space} can be written in the form
 
        \begin{equation}
                \partial_t \hat{ T_i} = F_i\per
        \end{equation}
        A forward Euler gives 
        \begin{equation}
            \label{ee_1}
            T_i^{n+1} = T_i^{n} + \Delta t F_i^n -\frac{\Delta t}{2}\partial_{tt}^2 T^{n}  +\frac{\Delta t^2}{6}\partial_{ttt}^3 T^{n} + \ord{(\Delta t^3)} \per
        \end{equation}
        Similarly, a backward Euler scheme gives
        \begin{equation}
            \label{ie_1}
            T_i^{n+1} = T_i^{n} + \Delta t F_i^{n+1}  + \frac{\Delta t}{2}\partial_{tt}^2 T^{n}  +\frac{\Delta t^2}{6}\partial_{ttt}^3 T^{n} + \ord{(\Delta t^3)} \per
        \end{equation}
        Both Euler schemes are first-order. The average of these schemes gives CN:
        \begin{equation}
            \label{cn_1}
            T_i^{n+1} = T_i^{n} + \frac{\Delta t}{2}\left(F_i^n + F_i^{n+1} \right) + \frac{\Delta t^2}{6}\partial_{ttt}^3 T^{n} +\ord{(\Delta t^3)}  \com
        \end{equation}
        which clearly has a truncation error $\ord{(\Delta t^2)}$; CN is second-order accurate. 


%\begin{figure}[p]
%\centerline{\epsfig{file=figs/stability_region_rk3.png,width=12cm}}
%\caption{Stability region for RK3 scheme used to march the diffusive-advective equation. The scheme is stable provided that $\delta$ and $r$ are such that  $\kappa' \Delta t = - 4 \, \delta \, \sin^2{\left(\tfrac{k \, \Delta x}{2}\right)}  -  \ii \, r \,\sin{\left(k\Delta x\right)}$ is within the yellowish region.}
%    \label{stability_rk3}
%\end{figure}


\end{enumerate}        


%\begin{lstlisting}[language=Python]
%\end{lstlisting}


\end{document}


